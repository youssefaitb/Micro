\documentclass[]{article}
\usepackage{lmodern}
\usepackage{amssymb,amsmath}
\usepackage{ifxetex,ifluatex}
\usepackage{fixltx2e} % provides \textsubscript
\ifnum 0\ifxetex 1\fi\ifluatex 1\fi=0 % if pdftex
  \usepackage[T1]{fontenc}
  \usepackage[utf8]{inputenc}
\else % if luatex or xelatex
  \ifxetex
    \usepackage{mathspec}
  \else
    \usepackage{fontspec}
  \fi
  \defaultfontfeatures{Ligatures=TeX,Scale=MatchLowercase}
\fi
% use upquote if available, for straight quotes in verbatim environments
\IfFileExists{upquote.sty}{\usepackage{upquote}}{}
% use microtype if available
\IfFileExists{microtype.sty}{%
\usepackage{microtype}
\UseMicrotypeSet[protrusion]{basicmath} % disable protrusion for tt fonts
}{}
\usepackage[margin=1in]{geometry}
\usepackage{hyperref}
\hypersetup{unicode=true,
            pdftitle={Data simualtion},
            pdfauthor={Y.A.B.},
            pdfborder={0 0 0},
            breaklinks=true}
\urlstyle{same}  % don't use monospace font for urls
\usepackage{color}
\usepackage{fancyvrb}
\newcommand{\VerbBar}{|}
\newcommand{\VERB}{\Verb[commandchars=\\\{\}]}
\DefineVerbatimEnvironment{Highlighting}{Verbatim}{commandchars=\\\{\}}
% Add ',fontsize=\small' for more characters per line
\usepackage{framed}
\definecolor{shadecolor}{RGB}{248,248,248}
\newenvironment{Shaded}{\begin{snugshade}}{\end{snugshade}}
\newcommand{\AlertTok}[1]{\textcolor[rgb]{0.94,0.16,0.16}{#1}}
\newcommand{\AnnotationTok}[1]{\textcolor[rgb]{0.56,0.35,0.01}{\textbf{\textit{#1}}}}
\newcommand{\AttributeTok}[1]{\textcolor[rgb]{0.77,0.63,0.00}{#1}}
\newcommand{\BaseNTok}[1]{\textcolor[rgb]{0.00,0.00,0.81}{#1}}
\newcommand{\BuiltInTok}[1]{#1}
\newcommand{\CharTok}[1]{\textcolor[rgb]{0.31,0.60,0.02}{#1}}
\newcommand{\CommentTok}[1]{\textcolor[rgb]{0.56,0.35,0.01}{\textit{#1}}}
\newcommand{\CommentVarTok}[1]{\textcolor[rgb]{0.56,0.35,0.01}{\textbf{\textit{#1}}}}
\newcommand{\ConstantTok}[1]{\textcolor[rgb]{0.00,0.00,0.00}{#1}}
\newcommand{\ControlFlowTok}[1]{\textcolor[rgb]{0.13,0.29,0.53}{\textbf{#1}}}
\newcommand{\DataTypeTok}[1]{\textcolor[rgb]{0.13,0.29,0.53}{#1}}
\newcommand{\DecValTok}[1]{\textcolor[rgb]{0.00,0.00,0.81}{#1}}
\newcommand{\DocumentationTok}[1]{\textcolor[rgb]{0.56,0.35,0.01}{\textbf{\textit{#1}}}}
\newcommand{\ErrorTok}[1]{\textcolor[rgb]{0.64,0.00,0.00}{\textbf{#1}}}
\newcommand{\ExtensionTok}[1]{#1}
\newcommand{\FloatTok}[1]{\textcolor[rgb]{0.00,0.00,0.81}{#1}}
\newcommand{\FunctionTok}[1]{\textcolor[rgb]{0.00,0.00,0.00}{#1}}
\newcommand{\ImportTok}[1]{#1}
\newcommand{\InformationTok}[1]{\textcolor[rgb]{0.56,0.35,0.01}{\textbf{\textit{#1}}}}
\newcommand{\KeywordTok}[1]{\textcolor[rgb]{0.13,0.29,0.53}{\textbf{#1}}}
\newcommand{\NormalTok}[1]{#1}
\newcommand{\OperatorTok}[1]{\textcolor[rgb]{0.81,0.36,0.00}{\textbf{#1}}}
\newcommand{\OtherTok}[1]{\textcolor[rgb]{0.56,0.35,0.01}{#1}}
\newcommand{\PreprocessorTok}[1]{\textcolor[rgb]{0.56,0.35,0.01}{\textit{#1}}}
\newcommand{\RegionMarkerTok}[1]{#1}
\newcommand{\SpecialCharTok}[1]{\textcolor[rgb]{0.00,0.00,0.00}{#1}}
\newcommand{\SpecialStringTok}[1]{\textcolor[rgb]{0.31,0.60,0.02}{#1}}
\newcommand{\StringTok}[1]{\textcolor[rgb]{0.31,0.60,0.02}{#1}}
\newcommand{\VariableTok}[1]{\textcolor[rgb]{0.00,0.00,0.00}{#1}}
\newcommand{\VerbatimStringTok}[1]{\textcolor[rgb]{0.31,0.60,0.02}{#1}}
\newcommand{\WarningTok}[1]{\textcolor[rgb]{0.56,0.35,0.01}{\textbf{\textit{#1}}}}
\usepackage{graphicx,grffile}
\makeatletter
\def\maxwidth{\ifdim\Gin@nat@width>\linewidth\linewidth\else\Gin@nat@width\fi}
\def\maxheight{\ifdim\Gin@nat@height>\textheight\textheight\else\Gin@nat@height\fi}
\makeatother
% Scale images if necessary, so that they will not overflow the page
% margins by default, and it is still possible to overwrite the defaults
% using explicit options in \includegraphics[width, height, ...]{}
\setkeys{Gin}{width=\maxwidth,height=\maxheight,keepaspectratio}
\IfFileExists{parskip.sty}{%
\usepackage{parskip}
}{% else
\setlength{\parindent}{0pt}
\setlength{\parskip}{6pt plus 2pt minus 1pt}
}
\setlength{\emergencystretch}{3em}  % prevent overfull lines
\providecommand{\tightlist}{%
  \setlength{\itemsep}{0pt}\setlength{\parskip}{0pt}}
\setcounter{secnumdepth}{0}
% Redefines (sub)paragraphs to behave more like sections
\ifx\paragraph\undefined\else
\let\oldparagraph\paragraph
\renewcommand{\paragraph}[1]{\oldparagraph{#1}\mbox{}}
\fi
\ifx\subparagraph\undefined\else
\let\oldsubparagraph\subparagraph
\renewcommand{\subparagraph}[1]{\oldsubparagraph{#1}\mbox{}}
\fi

%%% Use protect on footnotes to avoid problems with footnotes in titles
\let\rmarkdownfootnote\footnote%
\def\footnote{\protect\rmarkdownfootnote}

%%% Change title format to be more compact
\usepackage{titling}

% Create subtitle command for use in maketitle
\newcommand{\subtitle}[1]{
  \posttitle{
    \begin{center}\large#1\end{center}
    }
}

\setlength{\droptitle}{-2em}

  \title{Data simualtion}
    \pretitle{\vspace{\droptitle}\centering\huge}
  \posttitle{\par}
    \author{Y.A.B.}
    \preauthor{\centering\large\emph}
  \postauthor{\par}
      \predate{\centering\large\emph}
  \postdate{\par}
    \date{2/26/2019}


\begin{document}
\maketitle

\hypertarget{preamble}{%
\subsection{Preamble}\label{preamble}}

I started by trying to run simulations for my own research project (on
labor bargaining power in a sequential production model) but failed to
find any regression simulations that would be interesting at this stage
because the current identification that I have is very simple (it is not
the most elaborate component of the project) and I don't have any
interesting aspects to simulate.

So I stuck with the proposed framework..

\hypertarget{simulation}{%
\subsection{Simulation}\label{simulation}}

You can also embed plots, for example:

First we set up the data and main function.

\begin{Shaded}
\begin{Highlighting}[]
\NormalTok{lin_model <-}\StringTok{ }\ControlFlowTok{function}\NormalTok{(beta0, beta1, beta2, beta3, n, z, var_i)\{}
\NormalTok{  ind =}\StringTok{ }
\StringTok{    }\KeywordTok{data.frame}\NormalTok{(}
      \DataTypeTok{i =} \DecValTok{1}\OperatorTok{:}\NormalTok{n,}
      \DataTypeTok{ei =} \KeywordTok{rnorm}\NormalTok{(}\KeywordTok{length}\NormalTok{(}\DecValTok{1}\OperatorTok{:}\NormalTok{n), }\DataTypeTok{mean=}\DecValTok{0}\NormalTok{, }\DataTypeTok{sd=}\NormalTok{var_i),}
      \DataTypeTok{female =} \KeywordTok{c}\NormalTok{(}\KeywordTok{rep}\NormalTok{(}\DecValTok{1}\NormalTok{, z), }\KeywordTok{rep}\NormalTok{(}\DecValTok{0}\NormalTok{, }\DecValTok{10}\OperatorTok{-}\NormalTok{z)) }\OperatorTok\StringTok{ }\KeywordTok{rep}\NormalTok{(.,}\DecValTok{100}\NormalTok{),}
      \DataTypeTok{age =} \KeywordTok{rnorm}\NormalTok{(}\KeywordTok{length}\NormalTok{(}\DecValTok{1}\OperatorTok{:}\NormalTok{n), }\DataTypeTok{mean=}\DecValTok{46}\NormalTok{, }\DataTypeTok{sd=}\DecValTok{13}\NormalTok{)}
\NormalTok{    )}
  
\NormalTok{  ind <-}\StringTok{ }\NormalTok{ind }\OperatorTok\StringTok{ }\KeywordTok{mutate}\NormalTok{(}\DataTypeTok{height =}\NormalTok{ beta0 }\OperatorTok{+}\StringTok{ }\NormalTok{beta1 }\OperatorTok{*}\StringTok{ }\NormalTok{age }\OperatorTok{+}\StringTok{ }\NormalTok{beta2 }\OperatorTok{*}\StringTok{ }\NormalTok{female }\OperatorTok{+}\StringTok{ }\NormalTok{beta3 }\OperatorTok{*}\NormalTok{age}\OperatorTok{*}\NormalTok{female }\OperatorTok{+}\StringTok{ }\NormalTok{ei)}
  
\NormalTok{  model <-}\StringTok{ }\KeywordTok{lm}\NormalTok{(height }\OperatorTok{~}\StringTok{ }\NormalTok{female }\OperatorTok{+}\StringTok{ }\NormalTok{age }\OperatorTok{+}\StringTok{ }\NormalTok{female}\OperatorTok{*}\NormalTok{age, }\DataTypeTok{data=}\NormalTok{ind)}
\NormalTok{  betahat <-}\StringTok{ }\NormalTok{model}\OperatorTok{$}\NormalTok{coefficients[}\DecValTok{4}\NormalTok{]}
\NormalTok{  betahat <-}\StringTok{ }\KeywordTok{unname}\NormalTok{(betahat)}
  \KeywordTok{return}\NormalTok{(}\KeywordTok{list}\NormalTok{(}\StringTok{"betahat"}\NormalTok{=betahat))}
\NormalTok{\}}
\end{Highlighting}
\end{Shaded}

Then, we define the parameter space

\begin{Shaded}
\begin{Highlighting}[]
\CommentTok{# define fixed parameter grids:}
\NormalTok{  n_grid<-}\KeywordTok{c}\NormalTok{(}\DecValTok{1000}\NormalTok{)}
\NormalTok{  var_i_grid<-}\KeywordTok{c}\NormalTok{(}\DecValTok{1}\NormalTok{)}
\NormalTok{  beta0_grid<-}\KeywordTok{c}\NormalTok{(}\DecValTok{10}\NormalTok{)}
\NormalTok{  beta2_grid<-}\KeywordTok{c}\NormalTok{(}\OperatorTok{-}\DecValTok{6}\NormalTok{)}
\NormalTok{  beta3_grid<-}\KeywordTok{c}\NormalTok{(}\OperatorTok{-}\FloatTok{0.5}\NormalTok{)}
\NormalTok{  gamma_grid<-}\KeywordTok{c}\NormalTok{(}\DecValTok{0}\NormalTok{)}
\CommentTok{# beta_1 is the parameter that we will let vary to see if changes beta_3 so we specify a grid for beta1 }
\NormalTok{  beta1_grid<-}\KeywordTok{c}\NormalTok{(}\DecValTok{0}\NormalTok{,}\DecValTok{5}\NormalTok{,}\DecValTok{10}\NormalTok{)}
\NormalTok{  z_grid <-}\KeywordTok{c}\NormalTok{(}\DecValTok{3}\NormalTok{,}\DecValTok{5}\NormalTok{,}\DecValTok{8}\NormalTok{)}
  
\NormalTok{param_list=}\KeywordTok{list}\NormalTok{(}\StringTok{"n"}\NormalTok{=n_grid, }\StringTok{"z"}\NormalTok{=z_grid, }\StringTok{"var_i"}\NormalTok{=var_i_grid, }\StringTok{"beta0"}\NormalTok{=beta0_grid, }\StringTok{"beta1"}\NormalTok{=beta1_grid, }\StringTok{"beta2"}\NormalTok{=beta2_grid, }\StringTok{"beta3"}\NormalTok{=beta3_grid)}
\end{Highlighting}
\end{Shaded}

Running simulation

\begin{Shaded}
\begin{Highlighting}[]
\NormalTok{mc_result <-}\StringTok{ }\KeywordTok{MonteCarlo}\NormalTok{(}\DataTypeTok{func =}\NormalTok{ lin_model, }\DataTypeTok{nrep =} \DecValTok{500}\NormalTok{, }\DataTypeTok{param_list =}\NormalTok{ param_list)}
\end{Highlighting}
\end{Shaded}

\begin{verbatim}
## Grid of  9  parameter constellations to be evaluated. 
##  
## Progress: 
##  
## 
  |                                                                       
  |                                                                 |   0%
\end{verbatim}

\begin{verbatim}
## Warning in searchCommandline(parallel, cpus = cpus, type = type,
## socketHosts = socketHosts, : Unknown option on commandline:
## rmarkdown::render('/Users/yab/Documents/GitHub/Micro/
## Simulations.Rmd',~+~~+~encoding~+~
\end{verbatim}

\begin{verbatim}
## 
  |                                                                       
  |=======                                                          |  11%
\end{verbatim}

\begin{verbatim}
## Warning in searchCommandline(parallel, cpus = cpus, type = type,
## socketHosts = socketHosts, : Unknown option on commandline:
## rmarkdown::render('/Users/yab/Documents/GitHub/Micro/
## Simulations.Rmd',~+~~+~encoding~+~
\end{verbatim}

\begin{verbatim}
## 
  |                                                                       
  |==============                                                   |  22%
\end{verbatim}

\begin{verbatim}
## Warning in searchCommandline(parallel, cpus = cpus, type = type,
## socketHosts = socketHosts, : Unknown option on commandline:
## rmarkdown::render('/Users/yab/Documents/GitHub/Micro/
## Simulations.Rmd',~+~~+~encoding~+~
\end{verbatim}

\begin{verbatim}
## 
  |                                                                       
  |======================                                           |  33%
\end{verbatim}

\begin{verbatim}
## Warning in searchCommandline(parallel, cpus = cpus, type = type,
## socketHosts = socketHosts, : Unknown option on commandline:
## rmarkdown::render('/Users/yab/Documents/GitHub/Micro/
## Simulations.Rmd',~+~~+~encoding~+~
\end{verbatim}

\begin{verbatim}
## 
  |                                                                       
  |=============================                                    |  44%
\end{verbatim}

\begin{verbatim}
## Warning in searchCommandline(parallel, cpus = cpus, type = type,
## socketHosts = socketHosts, : Unknown option on commandline:
## rmarkdown::render('/Users/yab/Documents/GitHub/Micro/
## Simulations.Rmd',~+~~+~encoding~+~
\end{verbatim}

\begin{verbatim}
## 
  |                                                                       
  |====================================                             |  56%
\end{verbatim}

\begin{verbatim}
## Warning in searchCommandline(parallel, cpus = cpus, type = type,
## socketHosts = socketHosts, : Unknown option on commandline:
## rmarkdown::render('/Users/yab/Documents/GitHub/Micro/
## Simulations.Rmd',~+~~+~encoding~+~
\end{verbatim}

\begin{verbatim}
## 
  |                                                                       
  |===========================================                      |  67%
\end{verbatim}

\begin{verbatim}
## Warning in searchCommandline(parallel, cpus = cpus, type = type,
## socketHosts = socketHosts, : Unknown option on commandline:
## rmarkdown::render('/Users/yab/Documents/GitHub/Micro/
## Simulations.Rmd',~+~~+~encoding~+~
\end{verbatim}

\begin{verbatim}
## 
  |                                                                       
  |===================================================              |  78%
\end{verbatim}

\begin{verbatim}
## Warning in searchCommandline(parallel, cpus = cpus, type = type,
## socketHosts = socketHosts, : Unknown option on commandline:
## rmarkdown::render('/Users/yab/Documents/GitHub/Micro/
## Simulations.Rmd',~+~~+~encoding~+~
\end{verbatim}

\begin{verbatim}
## 
  |                                                                       
  |==========================================================       |  89%
\end{verbatim}

\begin{verbatim}
## Warning in searchCommandline(parallel, cpus = cpus, type = type,
## socketHosts = socketHosts, : Unknown option on commandline:
## rmarkdown::render('/Users/yab/Documents/GitHub/Micro/
## Simulations.Rmd',~+~~+~encoding~+~
\end{verbatim}

\begin{verbatim}
## 
  |                                                                       
  |=================================================================| 100%
## 
\end{verbatim}

Getting a summary of results

\begin{Shaded}
\begin{Highlighting}[]
\KeywordTok{summary}\NormalTok{(mc_result)}
\end{Highlighting}
\end{Shaded}

\begin{verbatim}
## Simulation of function: 
## 
## function(beta0, beta1, beta2, beta3, n, z, var_i){
##   ind = 
##     data.frame(
##       i = 1:n,
##       ei = rnorm(length(1:n), mean=0, sd=var_i),
##       female = c(rep(1, z), rep(0, 10-z)) %>% rep(.,100),
##       age = rnorm(length(1:n), mean=46, sd=13)
##     )
##   
##   ind <- ind %>% mutate(height = beta0 + beta1 * age + beta2 * female + beta3 *age*female + ei)
##   
##   model <- lm(height ~ female + age + female*age, data=ind)
##   betahat <- model$coefficients[4]
##   betahat <- unname(betahat)
##   return(list("betahat"=betahat))
## }
## <bytecode: 0x7fafd2f848e0>
## 
## Required time: 10.31 secs for nrep = 500  repetitions on 1 CPUs 
## 
## Parameter grid: 
## 
##  beta0 : 10 
##  beta1 : 0 5 10 
##  beta2 : -6 
##  beta3 : -0.5 
##      n : 1000 
##      z : 3 5 8 
##  var_i : 1 
## 
##  
## 1 output arrays of dimensions: 1 3 1 1 1 3 1 500
\end{verbatim}

\begin{Shaded}
\begin{Highlighting}[]
\NormalTok{  df.mc<-}\KeywordTok{MakeFrame}\NormalTok{(mc_result)}
  \KeywordTok{head}\NormalTok{(df.mc)}
\end{Highlighting}
\end{Shaded}

\begin{verbatim}
##   beta0 beta1 beta2 beta3    n z var_i    betahat
## 1    10     0    -6  -0.5 1000 3     1 -0.4985234
## 2    10     5    -6  -0.5 1000 3     1 -0.5014279
## 3    10    10    -6  -0.5 1000 3     1 -0.4983303
## 4    10     0    -6  -0.5 1000 5     1 -0.4944999
## 5    10     5    -6  -0.5 1000 5     1 -0.5055679
## 6    10    10    -6  -0.5 1000 5     1 -0.4999033
\end{verbatim}

For the three different levels of beta1, the point estimate of betahat
varies only marginally. We must however test the significance of the
difference to assert that there is no incidence.

Creating a graph

\begin{Shaded}
\begin{Highlighting}[]
\KeywordTok{class}\NormalTok{(df.mc}\OperatorTok{$}\NormalTok{z)}
\end{Highlighting}
\end{Shaded}

\begin{verbatim}
## [1] "numeric"
\end{verbatim}

\begin{Shaded}
\begin{Highlighting}[]
\NormalTok{df.mc}\OperatorTok{$}\NormalTok{z <-}\StringTok{ }\KeywordTok{as.factor}\NormalTok{(df.mc}\OperatorTok{$}\NormalTok{z)}
\KeywordTok{ggplot}\NormalTok{(df.mc, }\KeywordTok{aes}\NormalTok{(}\DataTypeTok{x=}\NormalTok{betahat, }\DataTypeTok{color=}\NormalTok{z)) }\OperatorTok{+}\StringTok{ }
\StringTok{  }\KeywordTok{geom_density}\NormalTok{() }\OperatorTok{+}
\StringTok{  }\KeywordTok{labs}\NormalTok{(}\DataTypeTok{color =} \KeywordTok{TeX}\NormalTok{(}\StringTok{"z="}\NormalTok{)) }\OperatorTok{+}\StringTok{ }
\StringTok{  }\KeywordTok{labs}\NormalTok{(}\DataTypeTok{title =} \KeywordTok{TeX}\NormalTok{(}\StringTok{"Distribution of the $}\CharTok{\textbackslash{}\textbackslash{}}\StringTok{beta_3$ estimate for different levels of gender mixity"}\NormalTok{),}
       \DataTypeTok{caption =} \KeywordTok{paste}\NormalTok{(}\StringTok{"Source: "}\NormalTok{,mc_result}\OperatorTok{$}\NormalTok{meta}\OperatorTok{$}\NormalTok{nrep,}\StringTok{" simulated samples of n="}\NormalTok{,n_grid,}\StringTok{", with true coefficient of interest = -0.5"}\NormalTok{)) }\OperatorTok{+}
\StringTok{  }\NormalTok{theme_gw}
\end{Highlighting}
\end{Shaded}

\includegraphics{Simulations_files/figure-latex/Figure1-1.pdf}

\begin{Shaded}
\begin{Highlighting}[]
\NormalTok{cbPalette <-}\StringTok{ }\KeywordTok{c}\NormalTok{(}\StringTok{"#999999"}\NormalTok{, }\StringTok{"#E69F00"}\NormalTok{, }\StringTok{"#56B4E9"}\NormalTok{, }\StringTok{"#009E73"}\NormalTok{, }\StringTok{"#F0E442"}\NormalTok{, }\StringTok{"#0072B2"}\NormalTok{, }\StringTok{"#D55E00"}\NormalTok{, }\StringTok{"#CC79A7"}\NormalTok{)}
  \KeywordTok{ggplot}\NormalTok{(df.mc, }\KeywordTok{aes}\NormalTok{(}\DataTypeTok{y=}\NormalTok{betahat, }\DataTypeTok{x=}\NormalTok{z, }\DataTypeTok{color=}\NormalTok{z)) }\OperatorTok{+}\StringTok{ }\KeywordTok{geom_boxplot}\NormalTok{() }\OperatorTok{+}
\StringTok{  }\KeywordTok{geom_hline}\NormalTok{(}\KeywordTok{aes}\NormalTok{(}\DataTypeTok{yintercept=}\NormalTok{mc_result}\OperatorTok{$}\NormalTok{param_list}\OperatorTok{$}\NormalTok{beta3), }\DataTypeTok{color=}\StringTok{"black"}\NormalTok{, }\DataTypeTok{linetype=}\StringTok{"solid"}\NormalTok{, }\DataTypeTok{size=}\NormalTok{.}\DecValTok{5}\NormalTok{) }\OperatorTok{+}
\StringTok{  }\KeywordTok{labs}\NormalTok{(}\DataTypeTok{color =} \KeywordTok{TeX}\NormalTok{(}\StringTok{"$z =$"}\NormalTok{)) }\OperatorTok{+}\StringTok{ }\KeywordTok{scale_fill_manual}\NormalTok{(}\DataTypeTok{values=}\NormalTok{cbPalette) }\OperatorTok{+}
\StringTok{  }\KeywordTok{labs}\NormalTok{(}\DataTypeTok{title =} \KeywordTok{TeX}\NormalTok{(}\StringTok{"Estimates of $}\CharTok{\textbackslash{}\textbackslash{}}\StringTok{beta_3$ for different proportions of gender mixity"}\NormalTok{),}
      \DataTypeTok{subtitle =} \KeywordTok{TeX}\NormalTok{(}\StringTok{"The estimates appear very comparable in terms of mean, but the precision varies a little"}\NormalTok{),}
       \DataTypeTok{caption =} \KeywordTok{paste}\NormalTok{(}\StringTok{"Source: "}\NormalTok{,mc_result}\OperatorTok{$}\NormalTok{meta}\OperatorTok{$}\NormalTok{nrep,}\StringTok{" simulated samples of n="}\NormalTok{,n_grid,}\StringTok{", with true coefficient of interest = -0.5"}\NormalTok{)) }\OperatorTok{+}
\StringTok{  }\NormalTok{theme_gw}
\end{Highlighting}
\end{Shaded}

\includegraphics{Simulations_files/figure-latex/Figure2-1.pdf}

This is not the graph I am trying to get, and I don't know why.

\#\#Part b: Letting gamma be different than 0.

\begin{Shaded}
\begin{Highlighting}[]
\NormalTok{lin_model_}\DecValTok{2}\NormalTok{ <-}\StringTok{ }\ControlFlowTok{function}\NormalTok{(beta0, beta1, beta2, beta3, gamma, n, z, var_i, var_x, mean_x)\{}
\NormalTok{  ind2 =}\StringTok{ }
\StringTok{    }\KeywordTok{data.frame}\NormalTok{(}
      \DataTypeTok{i =} \DecValTok{1}\OperatorTok{:}\NormalTok{n,}
      \DataTypeTok{ei =} \KeywordTok{rnorm}\NormalTok{(}\KeywordTok{length}\NormalTok{(}\DecValTok{1}\OperatorTok{:}\NormalTok{n), }\DataTypeTok{mean=}\DecValTok{0}\NormalTok{, }\DataTypeTok{sd=}\NormalTok{var_i),}
      \DataTypeTok{female =} \KeywordTok{c}\NormalTok{(}\KeywordTok{rep}\NormalTok{(}\DecValTok{1}\NormalTok{, z), }\KeywordTok{rep}\NormalTok{(}\DecValTok{0}\NormalTok{, }\DecValTok{10}\OperatorTok{-}\NormalTok{z)) }\OperatorTok\StringTok{ }\KeywordTok{rep}\NormalTok{(.,}\DecValTok{100}\NormalTok{),}
      \DataTypeTok{age =} \KeywordTok{rnorm}\NormalTok{(}\KeywordTok{length}\NormalTok{(}\DecValTok{1}\OperatorTok{:}\NormalTok{n), }\DataTypeTok{mean=}\DecValTok{46}\NormalTok{, }\DataTypeTok{sd=}\DecValTok{13}\NormalTok{),}
      \DataTypeTok{xi =} \KeywordTok{rnorm}\NormalTok{(}\KeywordTok{length}\NormalTok{(}\DecValTok{1}\OperatorTok{:}\NormalTok{n), }\DataTypeTok{mean=}\NormalTok{mean_x, }\DataTypeTok{sd=}\NormalTok{var_x)}
\NormalTok{    )}
  
\NormalTok{  ind2 <-}\StringTok{ }\NormalTok{ind2 }\OperatorTok\StringTok{ }\KeywordTok{mutate}\NormalTok{(}\DataTypeTok{height =}\NormalTok{ beta0 }\OperatorTok{+}\StringTok{ }\NormalTok{beta1 }\OperatorTok{*}\StringTok{ }\NormalTok{age }\OperatorTok{+}\StringTok{ }\NormalTok{beta2 }\OperatorTok{*}\StringTok{ }\NormalTok{female }\OperatorTok{+}\StringTok{ }\NormalTok{beta3 }\OperatorTok{*}\NormalTok{age}\OperatorTok{*}\NormalTok{female }\OperatorTok{+}\StringTok{ }\NormalTok{gamma}\OperatorTok{*}\NormalTok{xi }\OperatorTok{+}\StringTok{ }\NormalTok{ei)}
  
\NormalTok{  model}\FloatTok{.2}\NormalTok{ <-}\StringTok{ }\KeywordTok{lm}\NormalTok{(height }\OperatorTok{~}\StringTok{ }\NormalTok{age }\OperatorTok{+}\StringTok{ }\NormalTok{female }\OperatorTok{+}\StringTok{ }\NormalTok{female}\OperatorTok{*}\NormalTok{age }\OperatorTok{+}\StringTok{ }\NormalTok{xi, }\DataTypeTok{data=}\NormalTok{ind2)}
\NormalTok{  betahat <-}\StringTok{ }\NormalTok{model}\FloatTok{.2}\OperatorTok{$}\NormalTok{coefficients[}\DecValTok{5}\NormalTok{]}
\NormalTok{  betahat <-}\StringTok{ }\KeywordTok{unname}\NormalTok{(betahat)}
  \KeywordTok{return}\NormalTok{(}\KeywordTok{list}\NormalTok{(}\StringTok{"betahat"}\NormalTok{=betahat))}
\NormalTok{\}}
\end{Highlighting}
\end{Shaded}

\begin{Shaded}
\begin{Highlighting}[]
\CommentTok{# define fixed parameter grids:}
\NormalTok{  n_grid<-}\KeywordTok{c}\NormalTok{(}\DecValTok{1000}\NormalTok{)}
\NormalTok{  var_i_grid<-}\KeywordTok{c}\NormalTok{(}\DecValTok{1}\NormalTok{)}
\NormalTok{  beta0_grid<-}\KeywordTok{c}\NormalTok{(}\DecValTok{10}\NormalTok{)}
\NormalTok{  beta2_grid<-}\KeywordTok{c}\NormalTok{(}\OperatorTok{-}\DecValTok{6}\NormalTok{)}
\NormalTok{  beta3_grid<-}\KeywordTok{c}\NormalTok{(}\OperatorTok{-}\FloatTok{0.5}\NormalTok{)}
\NormalTok{  gamma_grid<-}\KeywordTok{c}\NormalTok{(}\DecValTok{0}\NormalTok{)}
\NormalTok{  beta1_grid<-}\KeywordTok{c}\NormalTok{(}\DecValTok{10}\NormalTok{)}
\NormalTok{  z_grid <-}\KeywordTok{c}\NormalTok{(}\DecValTok{5}\NormalTok{)}
\NormalTok{  mean_x_grid<-}\KeywordTok{c}\NormalTok{(}\DecValTok{0}\NormalTok{)}
\CommentTok{# changing properties of x }
\NormalTok{  var_x_grid<-}\KeywordTok{c}\NormalTok{(}\DecValTok{1}\NormalTok{,}\DecValTok{10}\NormalTok{,}\DecValTok{20}\NormalTok{)}
  

  
\NormalTok{param_list_}\DecValTok{2}\NormalTok{=}\KeywordTok{list}\NormalTok{(}\StringTok{"n"}\NormalTok{=n_grid, }\StringTok{"z"}\NormalTok{=z_grid, }\StringTok{"var_i"}\NormalTok{=var_i_grid, }\StringTok{"beta0"}\NormalTok{=beta0_grid, }\StringTok{"beta1"}\NormalTok{=beta1_grid, }\StringTok{"beta2"}\NormalTok{=beta2_grid, }\StringTok{"beta3"}\NormalTok{=beta3_grid, }\StringTok{"var_x"}\NormalTok{=var_x_grid, }\StringTok{"gamma"}\NormalTok{=gamma_grid, }\StringTok{"mean_x"}\NormalTok{=mean_x_grid)}
\end{Highlighting}
\end{Shaded}

\begin{Shaded}
\begin{Highlighting}[]
\NormalTok{mc_result_}\DecValTok{2}\NormalTok{ <-}\StringTok{ }\KeywordTok{MonteCarlo}\NormalTok{(}\DataTypeTok{func =}\NormalTok{ lin_model_}\DecValTok{2}\NormalTok{, }\DataTypeTok{nrep =} \DecValTok{500}\NormalTok{, }\DataTypeTok{param_list =}\NormalTok{ param_list_}\DecValTok{2}\NormalTok{)}
\end{Highlighting}
\end{Shaded}

\begin{verbatim}
## Grid of  3  parameter constellations to be evaluated. 
##  
## Progress: 
##  
## 
  |                                                                       
  |                                                                 |   0%
\end{verbatim}

\begin{verbatim}
## Warning in searchCommandline(parallel, cpus = cpus, type = type,
## socketHosts = socketHosts, : Unknown option on commandline:
## rmarkdown::render('/Users/yab/Documents/GitHub/Micro/
## Simulations.Rmd',~+~~+~encoding~+~
\end{verbatim}

\begin{verbatim}
## 
  |                                                                       
  |======================                                           |  33%
\end{verbatim}

\begin{verbatim}
## Warning in searchCommandline(parallel, cpus = cpus, type = type,
## socketHosts = socketHosts, : Unknown option on commandline:
## rmarkdown::render('/Users/yab/Documents/GitHub/Micro/
## Simulations.Rmd',~+~~+~encoding~+~
\end{verbatim}

\begin{verbatim}
## 
  |                                                                       
  |===========================================                      |  67%
\end{verbatim}

\begin{verbatim}
## Warning in searchCommandline(parallel, cpus = cpus, type = type,
## socketHosts = socketHosts, : Unknown option on commandline:
## rmarkdown::render('/Users/yab/Documents/GitHub/Micro/
## Simulations.Rmd',~+~~+~encoding~+~
\end{verbatim}

\begin{verbatim}
## 
  |                                                                       
  |=================================================================| 100%
## 
\end{verbatim}

\begin{Shaded}
\begin{Highlighting}[]
\KeywordTok{summary}\NormalTok{(mc_result_}\DecValTok{2}\NormalTok{)}
\end{Highlighting}
\end{Shaded}

\begin{verbatim}
## Simulation of function: 
## 
## function(beta0, beta1, beta2, beta3, gamma, n, z, var_i, var_x, mean_x){
##   ind2 = 
##     data.frame(
##       i = 1:n,
##       ei = rnorm(length(1:n), mean=0, sd=var_i),
##       female = c(rep(1, z), rep(0, 10-z)) %>% rep(.,100),
##       age = rnorm(length(1:n), mean=46, sd=13),
##       xi = rnorm(length(1:n), mean=mean_x, sd=var_x)
##     )
##   
##   ind2 <- ind2 %>% mutate(height = beta0 + beta1 * age + beta2 * female + beta3 *age*female + gamma*xi + ei)
##   
##   model.2 <- lm(height ~ age + female + female*age + xi, data=ind2)
##   betahat <- model.2$coefficients[5]
##   betahat <- unname(betahat)
##   return(list("betahat"=betahat))
## }
## <bytecode: 0x7fafd4ae89b0>
## 
## Required time: 4.07 secs for nrep = 500  repetitions on 1 CPUs 
## 
## Parameter grid: 
## 
##   beta0 : 10 
##   beta1 : 10 
##   beta2 : -6 
##   beta3 : -0.5 
##   gamma : 0 
##       n : 1000 
##       z : 5 
##   var_i : 1 
##   var_x : 1 10 20 
##  mean_x : 0 
## 
##  
## 1 output arrays of dimensions: 1 1 1 1 1 1 1 1 3 1 500
\end{verbatim}

\begin{Shaded}
\begin{Highlighting}[]
\NormalTok{  df.mc}\FloatTok{.2}\NormalTok{<-}\KeywordTok{MakeFrame}\NormalTok{(mc_result_}\DecValTok{2}\NormalTok{)}
  \KeywordTok{head}\NormalTok{(df.mc}\FloatTok{.2}\NormalTok{)}
\end{Highlighting}
\end{Shaded}

\begin{verbatim}
##   beta0 beta1 beta2 beta3 gamma    n z var_i var_x mean_x    betahat
## 1    10    10    -6  -0.5     0 1000 5     1     1      0 -0.4883851
## 2    10    10    -6  -0.5     0 1000 5     1    10      0 -0.4976448
## 3    10    10    -6  -0.5     0 1000 5     1    20      0 -0.4956476
## 4    10    10    -6  -0.5     0 1000 5     1     1      0 -0.5095058
## 5    10    10    -6  -0.5     0 1000 5     1    10      0 -0.5046935
## 6    10    10    -6  -0.5     0 1000 5     1    20      0 -0.4942778
\end{verbatim}

\begin{Shaded}
\begin{Highlighting}[]
\KeywordTok{class}\NormalTok{(df.mc}\FloatTok{.2}\OperatorTok{$}\NormalTok{var_x) }
\end{Highlighting}
\end{Shaded}

\begin{verbatim}
## [1] "numeric"
\end{verbatim}

\begin{Shaded}
\begin{Highlighting}[]
\NormalTok{df.mc}\FloatTok{.2}\OperatorTok{$}\NormalTok{var_x <-}\StringTok{ }\KeywordTok{as.factor}\NormalTok{(df.mc}\FloatTok{.2}\OperatorTok{$}\NormalTok{var_x)}
  \KeywordTok{ggplot}\NormalTok{(df.mc}\FloatTok{.2}\NormalTok{, }\KeywordTok{aes}\NormalTok{(}\DataTypeTok{y=}\NormalTok{betahat, }\DataTypeTok{x=}\NormalTok{var_x, }\DataTypeTok{color=}\NormalTok{var_x)) }\OperatorTok{+}\StringTok{ }\KeywordTok{geom_boxplot}\NormalTok{() }\OperatorTok{+}
\StringTok{  }\KeywordTok{geom_hline}\NormalTok{(}\KeywordTok{aes}\NormalTok{(}\DataTypeTok{yintercept=}\NormalTok{mc_result}\OperatorTok{$}\NormalTok{param_list}\OperatorTok{$}\NormalTok{beta3), }\DataTypeTok{color=}\StringTok{"black"}\NormalTok{, }\DataTypeTok{linetype=}\StringTok{"solid"}\NormalTok{, }\DataTypeTok{size=}\NormalTok{.}\DecValTok{5}\NormalTok{) }\OperatorTok{+}
\StringTok{  }\KeywordTok{labs}\NormalTok{(}\DataTypeTok{color =} \KeywordTok{TeX}\NormalTok{(}\StringTok{"$var_x =$"}\NormalTok{)) }\OperatorTok{+}\StringTok{ }\KeywordTok{scale_fill_manual}\NormalTok{(}\DataTypeTok{values=}\NormalTok{cbPalette) }\OperatorTok{+}
\StringTok{  }\KeywordTok{labs}\NormalTok{(}\DataTypeTok{title =} \KeywordTok{TeX}\NormalTok{(}\StringTok{"Estimation of $}\CharTok{\textbackslash{}\textbackslash{}}\StringTok{beta_3$ for different variances of X"}\NormalTok{),}
      \DataTypeTok{subtitle =} \KeywordTok{TeX}\NormalTok{(}\StringTok{"$}\CharTok{\textbackslash{}\textbackslash{}}\StringTok{gamma = 0$ changes in properties of X are not relevant to estimation of $}\CharTok{\textbackslash{}\textbackslash{}}\StringTok{beta_3$"}\NormalTok{),}
       \DataTypeTok{caption =} \KeywordTok{paste}\NormalTok{(}\StringTok{"Source: "}\NormalTok{,mc_result}\OperatorTok{$}\NormalTok{meta}\OperatorTok{$}\NormalTok{nrep,}\StringTok{" simulated samples of n="}\NormalTok{,n_grid,}\StringTok{", with true coefficient of interest = -0.5"}\NormalTok{)) }\OperatorTok{+}
\StringTok{  }\NormalTok{theme_gw}
\end{Highlighting}
\end{Shaded}

\includegraphics{Simulations_files/figure-latex/Figure3-1.pdf}

This result is expectable because X is here built to be irrelevant to
the model, therefore no bias is introduced.


\end{document}
